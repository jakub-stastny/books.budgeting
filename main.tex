\documentclass[a5paper,10pt]{book}
\usepackage[utf8]{inputenc}
\usepackage{graphicx}

\usepackage[pdftex,
            pdfauthor={Jakub Šťastný},
            pdftitle={Budgetting made simple},
            pdfsubject={The Subject},
            pdfkeywords={budget, budgetting, personal finance},
            pdfproducer={LaTeX},
            pdfcreator={PDFLaTeX}]{hyperref}

\begin{document}

% Cover page
\begin{titlepage}
%\noindent
%\thispagestyle{empty}
%\setcounter{page}{0}
\includegraphics[width=\pdfpagewidth,height=\pdfpageheight]{"assets/cover"}
\end{titlepage}

\begin{figure}
  \centering
  \fbox{\includegraphics[width=0.5\textwidth]{"assets/balance"}} 
\end{figure}

\section{Introduction}
\date
Doesn't matter how much you make if your back pocket has a hole in it.

\quote{Be a good treasurer}, my grandfather used to say.

I saw the matter differently \quote{Why bother? So much hassle with planning, having a budget and all that. I'm making a lot of money, so why bother anyways.}

And I was making good money.

But whatever I made never stuck with me for long. Trip to Asia, bespoke suit ... and my bank account empty again. Then I noticed that my grandfather was right.

\section{OK let's budget, but how?}

I tried many apps and ways to get organised. Nothing worked – at least until I decided to strip everything to bare bones: cash and weekly budgets.

\subsection{Cards and budgets}
\subsection{Months and budgets}

\section{Getting started}

The good news is that all you need is pen and paper, 3 envelopes and some index cards.

\subsection{The three envelopes}

\subsubsection{Current week}

\subsubsection{Savings}

\subsubsection{Coming weeks}

\end{document}